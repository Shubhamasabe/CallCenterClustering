\section{Conclusions} 
\subsection{Conclusions}
\paragraph{}The performance of the tool and the quality of the obtained results was improved using a specific cartridge to Portuguese language. This facility contains particular properties that are very helpful. In accordance with the obtained results, it must be stressed that the sample of data, as well as the activities carried through the pre-processing step and clustering process are of fundamental importance to the knowledge extraction. In our experiments the traditional categorization process built during the company practice was confirmed, and could be validated. Although promising results have been achieved in this work, there are some issues that can be further investigated by the call center experts. But we can conclude that now is possible to implement an automatic classification system to online monitor the service quality. Finally it can be concluded that the obtained text mining results can be used in the power electric industry to: Understand the needs of customers while accessing preferred customers;
\begin{itemize}
	

\item  Provide service menus for locking in preferred customers, such as offering discount charge menus and added-value services;
\item  Analyze how both electric power income and related income change, and investigate the contents of the services;
\item  Measure whether customers offered services are satisfied with those services and reflect the results in the service menu planning.
\end{itemize}
\subsection{Future Work}
\paragraph{}Thefeature work also added customers’ bills invoicing, probably one of the problems that more receive calls in all the existing topics in a company from the power electric industry.
\subsection{Applications}

 

	 Successful centers use advanced call center analytics software to monitor and review performance, not only from a customer lens, also from the employee’s perspective, as well as a business-owner lens.\\
	
	Each of these approaches offers its own advantages and together satisfies each angle. The key to choosing the correct analytics combination lies in understanding the approaches, and how they can be used to improve your call center. Here are the six most common approaches to analytics:\\
	\begin{enumerate}
\item	\textbf{Call Center Speech Analytics}\\ Speech Analytics is a fairly new and relatively rare field, but one that many early adopters are finding significant success with. Using a team of analysts to monitor calls in real time, a company can unearth inefficiencies in their current model, and make process improvements, such as moving to a call script, or developing systems for call center agents to utilize in order to achieve the desired call outcome.
\item	\textbf{Call Center Text Analytics}\\ The last several years have seen an an explosion in the social media universe, and most forward thinking companies have developed a brand presence online. This paradigm shift has rendered text analytics ever more important, as we are no longer communicating with our customers through written documents, but also through email, secure messaging, Facebook, Twitter, and other text-centered media. Text analytics can review and monitor not only the messages sent to customers, but also the message they are sending to the company. This is vital in seeing any potential issues through the customer lens.
\item 	\textbf{Predictive Analytics}\\ The modern predictive analysis engine is an invaluable tool in the call center environment. Using in-depth review of past performance in areas as diverse as call volume, service level, handle time, and customer satisfaction, predictive analysis makes it possible to apply past solutions to upcoming problems. How many agents will we need staffed on Christmas Day? How will your new product rollout affect call volume on weekends? What will this change to your fee structure do to your customer satisfaction score? By analyzing past results, companies can plan and strategize for the future.
	
\end{enumerate}
\newpage
