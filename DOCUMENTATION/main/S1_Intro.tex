\section{Introduction}
\paragraph{}Many companies who give telephonic attendance services for its customers need solutions guided to the information technology that make possible the register of their customers contacts, the aid and support to the attendants. For such, new access technologies to the knowledge bases and flow of information management can increase the productivity of the attendants and, as consequence, improve the quality of the given services. Many times, the attendance can be made by the Internet, bringing some advantages such as: costs reduction with attendance, user satisfaction increasing and knowledge of the customers interests, gotten through the automatic storage of the typed information. Text mining tools directed to knowledge bases creation, support and operation, and for attendance centers are applicable for the Web cases email or telephone attendance. The technologies presented in these tools make possible to create knowledge bases for queries through the search of one or more words in documents, allow to combine resources of search in texts with resources of interaction with support (attendance it the Web: email, chats, virtual rooms, etc), and make possible the creation of knowledge bases for queries in natural language. Thus, the objective of this work is to show some methods of text mining applied in call center’s databases, especially in customers’ complaints of a electrical energy company, where the results serve to show that techniques are extremely useful during the company services evaluation processes, as also to be used in decision taking. For this, a text mining tool called Insight Discoverer Clusterer by Temis was used, that contains the documents grouping functionality or clustering, allowing the classification of the same ones. With this study we have created conditions that can to evidence problems that are occurring with frequency, as for example problems of invoicing, financial income, supply and attendance quality, etc, who are gotten by the analysis of the results.

\subsection{\textbf{Overview}}
The main objective is to identify new and useful knowledge, based on customers’ claims. Through the information agreement, it will be possible to identify better ways to help the customer, increasing their satisfaction with company services as well as supplying the call center staff and other related areas with a set of procedures and information concerning the most common customer’s questions.
\subsection{\textbf{Motivation}}
In today's world , many customer send request to call center there is a lot of waiting for thousands of people at a time ,to make this quick and efficient we are going to work on this project
This was the problem statement given by TCS in their hackathone

\subsection{\textbf{Problem Definition and Objectives }}
In today's global marketplace, many companies have turned to a call center model to assist, streamline and maximize customer service and sales needs at scale. With one eye focused on providing excellent support and the other on efficiency, an ideal call center needs to strike that perfect balance of care and resources. While call centers require an increase in overhead, they also sacrifice valuable "face to face" customer interaction. The current state of call center analytics provides employers the ability to improve service quality and doing so with the bottom line in mind.
\subsection{\textbf{Project Scope \& Limitations }}
Successful centers use advanced call center analytics software to monitor and review performance, not only from a customer lens, also from the employee’s perspective, as well as a business-owner lens.\\

Each of these approaches offers its own advantages and together satisfies each angle. The key to choosing the correct analytics combination lies in understanding the approaches, and how they can be used to improve your call center. Here are the six most common approaches to analytics.


\subsection{\textbf{Methodologies of Problem Solving  }}
Our model was using the TF*IDF algorithm is used for  weighing a keyword in any content and assign the importance to that keyword based on the number of times it appears in the document.The value of K defines the Number of clusters that is the 5 categories we are choosing.



\newpage
